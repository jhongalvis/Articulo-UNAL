\section{Conclusiones}

Este estudio confirma que el ajuste persona–trabajo es un predictor robusto de la intención de renuncia entre tecnólogos colombianos del sector eléctrico, electrónico y electromecánico. Al diferenciar entre la congruencia de habilidades técnicas (\emph{Skill--Job Fit}) y la compatibilidad socio–cultural (\emph{Personality--Job Fit}), se revelaron rutas de influencia diferenciadas: mientras que la congruencia técnica incrementa la satisfacción laboral y, a través de ella, reduce la intención de renunciar, la compatibilidad socio–cultural fortalece la percepción de apoyo organizacional percibido (POS) y reduce la renuncia mediante esta ruta. Este hallazgo subraya la necesidad de conceptualizar el ajuste persona–trabajo como un constructo multidimensional y de considerar simultáneamente los mecanismos afectivos a través de los cuales opera.

En términos de magnitud, los resultados del modelo parsimonioso (15 ítems) mostraron que el ajuste técnico ejerce un efecto positivo significativo sobre la satisfacción ($\beta=0{,}39$) y el POS ($\beta=0{,}20$), generando un efecto indirecto total de $\beta_{\text{ind}}=-0{,}24$ sobre la intención de renuncia. Por su parte, el ajuste socio–cultural fue el predictor directo más fuerte del POS ($\beta=0{,}42$) pero no predijo la satisfacción, operando exclusivamente a través del POS para reducir la renuncia ($\beta_{\text{ind}}=-0{,}19$). Ambas actitudes laborales influyeron negativamente en la intención de renuncia con magnitudes similares ($\beta_{\text{satisfacción}}=-0{,}39$; $\beta_{\text{POS}}=-0{,}44$), confirmando la centralidad de estos mediadores afectivos. En conjunto, el modelo explicó un 45\,\% de la varianza de la intención de renuncia, superando los promedios meta–analíticos reportados en la literatura (30–35\,\%) e indicando la relevancia de este enfoque en contextos de alta competitividad técnica.

Las contribuciones de este trabajo son múltiples. Primero, ratifica la multidimensionalidad del ajuste persona–trabajo y demuestra que sus dimensiones predicen resultados organizacionales de manera diferenciada, lo que refuerza la pertinencia de desagregar el P–J Fit en sus componentes conceptuales \cite{lauver_distinguishing_2001,Cable2002}. Segundo, identifica al apoyo organizacional percibido como mediador crítico y con frecuencia dominante en la relación entre el ajuste y la intención de renuncia, extendiendo modelos previos que tradicionalmente privilegiaban la satisfacción como único mediador \cite{tett_job_1993,griffeth_meta_2000}. En ocupaciones donde las competencias técnicas son escasas, el reconocimiento y cuidado organizacional emergen como recursos socioemocionales decisivos para retener talento. Tercero, la integración de diversas metodologías psicométricas—Teoría Clásica, análisis factorial exploratorio y confirmatorio, y Teoría de Respuesta al Ítem—demuestra que la depuración basada en TRI permite construir un instrumento breve (15 ítems, $\approx 5$ minutos) con excelentes propiedades psicométricas (cargas $\lambda\geq0{,}73$, CR$\geq0{,}91$, AVE$\geq0{,}67$, CFI/TLI$\approx0,99$) y mejora sustancialmente el ajuste de los modelos SEM sin sacrificar validez ni fiabilidad. Este instrumento puede emplearse en futuras investigaciones y evaluaciones organizacionales para diagnosticar riesgos de rotación de manera rápida y precisa.

Desde la perspectiva de gestión, las organizaciones deberían diseñar procesos de selección y desarrollo que optimicen el ajuste técnico de sus colaboradores (por ejemplo, análisis de puestos, entrevistas basadas en competencias, programas de reskilling/upskilling), ya que el ajuste de habilidades no solo mejora el desempeño y la satisfacción sino que también fortalece la percepción de apoyo al reconocer la contribución del empleado. Paralelamente, deben promover entornos de apoyo que satisfagan las necesidades socioemocionales de los tecnólogos mediante prácticas como la retroalimentación constructiva, el reconocimiento formal de logros, la inversión en desarrollo profesional y la creación de un clima de justicia y reciprocidad \cite{rhoades_perceived_2002}. La evidencia demuestra que el POS es un mediador dominante de la retención en contextos donde la empleabilidad técnica es alta y la movilidad laboral es factible.

Finalmente, aunque el estudio ofrece resultados sólidos, presenta limitaciones que deben considerarse. El diseño transversal y el muestreo no probabilístico impiden establecer causalidad y limitan la generalización; se sugiere replicar los hallazgos con diseños longitudinales, muestras probabilísticas y en otros sectores productivos para consolidar la evidencia. También es recomendable explorar moderadores como la edad, la experiencia laboral, las condiciones del mercado de trabajo y otros tipos de ajuste (persona–organización, persona–equipo) para entender mejor la complejidad de la retención. La aplicación del instrumento abreviado en contextos comparativos internacionales podría aportar evidencia sobre la validez transcultural del modelo.

En conclusión, distinguir entre ajuste técnico y socio–cultural y comprender sus mecanismos afectivos de influencia ofrece a las organizaciones herramientas valiosas para retener talento crítico. La reducción del instrumento a 15 ítems facilita su implementación en la práctica, y la replicación de estos resultados en otras poblaciones consolidará las bases para desarrollar estrategias de gestión de recursos humanos más precisas y eficaces en la prevención de la rotación voluntaria.

