% Sección de título, autores, resúmenes y palabras clave

\title[maintitle = Efecto del ajuste persona--trabajo en la satisfacción laboral y la intención de renuncia: rol mediador del apoyo organizacional percibido en tecnólogos colombianos,
       secondtitle = Effect of person--job fit on job satisfaction and turnover intention: mediating role of perceived organizational support among Colombian technologists,
       shorttitle = Person--job fit and turnover intention]{ }

\begin{authors}
\author[firstname = Jhon Jairo,
        surname = Galvis López,
        numberinstitution = 1,
        affiliation = {Estudiante Maestría en Estadística},
        email = jhon.galvis@usantotomas.edu.co]
\author[firstname = Juan Carlos,
        surname = Rubriche,
        numberinstitution = 1,
        affiliation = {Director de tesis},
        email = juanrubriche@usantotomas.edu.co]
\end{authors}

\begin{institutions}
     \institute[subdivision = Facultad de Estadística,
                institution = Universidad Santo Tomás,
                city = Bogotá,
                country = Colombia]
\end{institutions}

\begin{mainabstract}
Este estudio examina los mecanismos mediante los cuales el ajuste persona--trabajo (Person--Job Fit, P--J Fit) influye sobre la intención de renuncia en tecnólogos colombianos, diferenciando entre ajuste habilidad--trabajo (Skill--Job Fit) y ajuste personalidad--trabajo (Personality--Job Fit), y proponiendo que satisfacción laboral y apoyo organizacional percibido (Perceived Organizational Support, POS) operan como mediadores paralelos. Se empleó un diseño transversal con muestreo no probabilístico por conveniencia (\(N=452\) tecnólogos de sectores eléctrico, electrónico y electromecánico), aplicando análisis psicométricos: Teoría Clásica de Tests, Análisis Factorial Exploratorio y Confirmatorio, Teoría de Respuesta al Ítem y modelos de ecuaciones estructurales con estimador WLSMV para datos ordinales. La depuración basada en Teoría de Respuesta al Ítem redujo el instrumento de 30 a 15 ítems, mejorando el ajuste global (CFI/TLI \(\approx .99\); RMSEA \(= .049\)). Los resultados confirman que el ajuste habilidad--trabajo predice positivamente la satisfacción laboral (\(\beta=.394,\ p<.001\)) y el apoyo organizacional percibido (\(\beta=.202,\ p=.003\)), mientras que el ajuste personalidad--trabajo predice significativamente el POS (\(\beta=.423,\ p<.001\)) pero no la satisfacción (\(\beta=.120,\ p=.111\)). Ambas actitudes reducen significativamente la intención de renuncia (Satisfacción: \(\beta=-.390\); POS: \(\beta=-.437\), ambos \(p<.001\)), explicando conjuntamente el 43\% de su varianza. Los efectos indirectos del ajuste habilidad--trabajo sobre la intención de renuncia fueron \(\beta_{\text{ind}}=-.242\) (\(p<.001\)), operando fundamentalmente vía satisfacción (64\%), mientras que el ajuste personalidad--trabajo operó vía POS (\(\beta_{\text{ind}}=-.185\), \(p<.001\)). Los hallazgos resaltan la importancia de optimizar el ajuste técnico y visibilizar el apoyo organizacional para reducir la rotación voluntaria en ocupaciones tecnológicas.
\keywords{ajuste persona--trabajo; satisfacción laboral; intención de renuncia; apoyo organizacional percibido; tecnólogos; modelos de ecuaciones estructurales}
\end{mainabstract}

\begin{secondaryabstract}
This study examines the mechanisms by which person--job fit influences turnover intention among Colombian technologists, distinguishing skill--job fit and personality--job fit and proposing job satisfaction and perceived organizational support (POS) as parallel mediators. A cross-sectional design with convenience sampling was used (\(N=452\) technologists in the electrical, electronic and electromechanical sectors), applying psychometric analyses: classical test theory, exploratory and confirmatory factor analysis, item response theory and structural equation modeling (WLSMV) for ordinal data. Item response theory trimming reduced the instrument from 30 to 15 items, improving global fit (CFI/TLI \(\approx .99\); RMSEA \(= .049\)). Results confirm that skill--job fit positively predicts job satisfaction (\(\beta=.394,\ p<.001\)) and perceived organizational support (\(\beta=.202,\ p=.003\)), whereas personality--job fit significantly predicts POS (\(\beta=.423,\ p<.001\)) but not satisfaction (\(\beta=.120,\ p=.111\)). Both attitudes significantly reduce turnover intention (satisfaction: \(\beta=-.390\); POS: \(\beta=-.437\), both \(p<.001\)), jointly explaining 43\% of its variance. The indirect effects of skill--job fit on turnover intention were \(\beta_{\text{ind}}=-.242\) (\(p<.001\)), mainly via satisfaction (64\%), while personality--job fit acted through POS (\(\beta_{\text{ind}}=-.185\), \(p<.001\)). The findings highlight the importance of optimizing technical fit and signalling organizational support to reduce voluntary turnover in technological occupations.
\keywords{person--job fit; job satisfaction; turnover intention; perceived organizational support; technologists; structural equation modeling}
\end{secondaryabstract}