\section{Métodos}

\subsection{Diseño y participantes}
Se adoptó un diseño cuantitativo, explicativo, no experimental y de corte transversal con estrategia de modelamiento de ecuaciones estructurales (SEM) para contrastar las hipótesis (H1–H7). La investigación se centró en tecnólogos titulados en Electricidad, Electrónica, Electromecánica y áreas afines en Colombia, colectivos clave por su papel de bisagra entre las tareas operativas y de ingeniería en sectores sometidos a rápidas transformaciones tecnológicas.

El marco muestral provino de la base de datos del Consejo Profesional Nacional de Tecnólogos en Electricidad (CONALTEL). Se envió por correo electrónico una invitación a participar, que explicaba los objetivos académicos, la naturaleza voluntaria de la participación y la garantía de confidencialidad. De 487 cuestionarios iniciados, se obtuvieron 452 respuestas completas (tasa de completitud del 92,8 \%), tras excluir 35 casos con más del 20 \% de ítems sin responder. El muestreo fue no probabilístico por conveniencia, pero se fijaron criterios de inclusión para asegurar la pertinencia de los datos: (a) estar laboralmente activo en un puesto relacionado con la formación técnica; (b) contar con al menos seis meses de antigüedad en el cargo actual; y (c) aceptar el consentimiento informado. Por política de anonimato, no se recolectaron datos demográficos que pudieran identificar a los participantes (edad, género, ubicación, empleador), por lo que no se dispone de estadísticas descriptivas al respecto. La focalización en tecnólogos responde a su importancia en cadenas de valor sensibles a la continuidad operativa y a su frecuente subrepresentación en la literatura organizacional.

\subsection{Instrumentos}
El cuestionario incluyó cinco escalas tipo Likert de siete puntos (1 = \emph{Totalmente en desacuerdo}, 7 = \emph{Totalmente de acuerdo}), adaptadas al español mediante un proceso de traducción y ajuste terminológico, seguido de una prueba piloto con diez tecnólogos para asegurar la claridad semántica. Cada escala estaba compuesta inicialmente por seis ítems, y las puntuaciones altas indican mayor ajuste, satisfacción, apoyo percibido o intención de renuncia. Se emplearon las siguientes medidas:

\begin{itemize}
  \item \textbf{Skill--Job Fit} y \textbf{Personality--Job Fit}. Cada dimensión se evaluó con seis ítems adaptados de \cite{lauver_distinguishing_2001}, que captan la correspondencia entre las habilidades técnicas y las demandas del puesto, y la congruencia entre los valores/rasgos de personalidad y la cultura laboral. Ejemplos de ítems son: “Mis habilidades técnicas coinciden bien con las demandas de mi puesto” y “Los valores que promueve mi trabajo son compatibles con mis valores personales”.
  \item \textbf{Satisfacción laboral}. Se utilizó la subescala global de seis ítems del Michigan Organizational Assessment Questionnaire (MOAQ–JSS) de \cite{bowling_meta_2008}, que evalúa la satisfacción global con el trabajo. Ejemplo: “En términos generales, estoy satisfecho(a) con mi trabajo”.
  \item \textbf{Intención de renuncia}. Se aplicaron seis ítems de \cite{miller_evaluation_1979} para medir la frecuencia con que el empleado piensa en renunciar y la probabilidad de buscar un nuevo empleo (“A menudo pienso en renunciar a mi empleo”; “Es muy probable que busque un nuevo trabajo en el próximo año”).
  \item \textbf{Apoyo organizacional percibido (POS)}. Se usó la versión abreviada de seis ítems del Survey of Perceived Organizational Support de \cite{eisenberger_perceived_1997}, que evalúa el grado en que la organización valora las contribuciones y se preocupa por el bienestar del empleado.
\end{itemize}

Las escalas presentaron alta fiabilidad en el análisis preliminar (alfa ordinal y omega > 0,85), y la validez convergente y discriminante quedó confirmada mediante análisis factorial exploratorio y confirmatorio (véase Tabla \ref{tab:fiabilidad}). La traducción y adaptación cultural siguió un proceso de tres etapas: traducción directa por el investigador principal, ajuste terminológico al contexto colombiano y una prueba piloto que no identificó problemas de comprensión.

\subsection{Procedimiento}
El cuestionario se distribuyó mediante un formulario en línea en una plataforma segura. La primera pantalla contenía el consentimiento informado digital, que detallaba los objetivos académicos, la voluntariedad de la participación, el derecho a retirarse sin consecuencias y la garantía de anonimato y confidencialidad. Sólo los participantes que aceptaron expresamente continuaron al cuestionario.

Tras la recolección de los datos, se siguieron las siguientes etapas analíticas:

\begin{enumerate}
  \item Se calcularon estadísticos descriptivos (medias, desviaciones estándar, asimetría, curtosis) y fiabilidades (alfa ordinal, omega, confiabilidad compuesta) para cada escala mediante Teoría Clásica de los Tests.
  \item Se realizó un Análisis Factorial Exploratorio (AFE) sobre la matriz de correlaciones policóricas de un tercio de la muestra (\(n_{\text{AFE}}\approx 150\)), utilizando el método de mínimos residuales y rotación oblimin para identificar la estructura latente inicial.
  \item Se llevó a cabo un Análisis Factorial Confirmatorio (AFC) con estimador WLSMV en la submuestra restante (\(n_{\text{AFC}}\approx 302\)) para validar un modelo de cinco factores. Se examinaron las cargas estandarizadas (λ), la fiabilidad compuesta (CR), la varianza media extraída (AVE) y la validez discriminante mediante el criterio de Fornell–Larcker y el índice HTMT.
  \item Se aplicó un modelo de Teoría de Respuesta al Ítem (modelo de respuestas graduadas) para seleccionar los tres ítems con mayor poder discriminativo en cada escala, reduciendo el instrumento de 30 a 15 ítems sin pérdida de representatividad.
  \item Finalmente, se estimó un modelo de ecuaciones estructurales (SEM) con las cinco variables latentes definidas por los 15 ítems seleccionados para probar las hipótesis sobre efectos directos e indirectos entre el ajuste persona–trabajo, el apoyo organizacional percibido, la satisfacción laboral y la intención de renuncia. Se reportaron los coeficientes estandarizados, los efectos indirectos mediante \emph{bootstrap} y la varianza explicada (\(R^2\)) para cada variable endógena; el ajuste global se evaluó con los índices $\chi^2$ escalado, CFI, TLI, RMSEA y SRMR.
\end{enumerate}

Todos los análisis se realizaron en \textbf{R} 4.3.1 \cite{r_core_team_2023} utilizando paquetes de código abierto: \texttt{psych} para las estadísticas descriptivas y matrices policóricas, \texttt{lavaan} para el SEM \cite{rosseel_lavaan_2012}, \texttt{semTools} para el cálculo de CR, AVE y HTMT, \texttt{ltm} y \texttt{mirt} para el modelo de respuesta graduada, y \texttt{tidyverse} para la gestión y visualización de datos. Este flujo analítico asegura reproducibilidad y transparencia metodológica.

\subsection*{Consideraciones éticas}
El estudio se diseñó conforme a la Declaración de Helsinki y la Resolución 8430 de 1993 del Ministerio de Salud de Colombia. Como la investigación sólo implicó la recolección anónima de percepciones laborales sin intervención experimental, se clasificó como de \emph{riesgo mínimo}. Las medidas éticas adoptadas incluyeron: (a) obtención del consentimiento informado digital en la primera pantalla del cuestionario; (b) anonimato garantizado al no recolectar datos identificables; (c) confidencialidad mediante almacenamiento seguro y acceso restringido; (d) derecho a la información y a la revocación de los datos conforme a la Ley 1581 de 2012; y (e) autorización institucional de CONALTEL para contactar a los tecnólogos, condicionada al uso académico exclusivo de los datos.
