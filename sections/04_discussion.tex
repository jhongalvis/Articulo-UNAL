\section{Discusión}

En esta sección se interpretan los hallazgos a la luz del marco teórico del ajuste persona–trabajo (P–J Fit) y se examinan sus contribuciones científicas, implicaciones prácticas y limitaciones. El estudio diferenció entre dos dimensiones del ajuste (habilidades técnicas y compatibilidad socio–cultural) y evaluó su influencia sobre la intención de renuncia a través de dos mediadores afectivos (satisfacción laboral y apoyo organizacional percibido), utilizando un modelo de ecuaciones estructurales parsimonioso validado mediante múltiples enfoques psicométricos.

\subsection{Hallazgos principales y relación con la teoría}

El modelo SEM parsimonioso (15 indicadores) corroboró que el ajuste persona–trabajo es un determinante crucial de la retención de tecnólogos colombianos, explicando cerca del 45\,\% de la varianza en intención de renuncia. No obstante, reveló asimetrías importantes en cómo operan sus dimensiones: el \emph{Skill--Job Fit} fue un predictor positivo significativo de la satisfacción laboral ($\beta=0.39$) y del apoyo organizacional percibido ($\beta=0.20$), mientras que el \emph{Personality--Job Fit} predijo fuertemente el POS ($\beta=0.42$) pero no la satisfacción ($\beta=0.12$, n.s.). Estos resultados respaldan la teoría Needs–Supplies según la cual la congruencia entre las competencias del individuo y los requisitos del puesto genera sensaciones de dominio y autoeficacia que alimentan la satisfacción laboral y la reciprocidad organizacional. A su vez, el ajuste socio–cultural opera fundamentalmente sobre las percepciones de la relación empleado–organización, fortaleciendo el sentido de apoyo y reconocimiento más que la satisfacción intrínseca con el contenido del trabajo.

Los análisis de mediación mostraron que el efecto total de \emph{Skill--Job Fit} sobre la intención de renuncia se descompone en un 64\,\% a través de la satisfacción laboral y un 36\,\% a través del POS, confirmando que la satisfacción es un antecedente clave de la rotación voluntaria. En cambio, el efecto de \emph{Personality--Job Fit} se transmitió exclusivamente a través del POS, subrayando que la compatibilidad socio–cultural alimenta la reciprocidad organizacional más que las evaluaciones del trabajo en sí. Ambos mediadores ejercieron efectos directos negativos y de magnitud similar sobre la intención de renuncia ($\beta_{\text{satisfacción}}=-0.39$; $\beta_{\text{POS}}=-0.44$), siendo el POS ligeramente más fuerte, lo que apunta a la importancia de los recursos socioemocionales en contextos de alta competencia técnica.

\subsection{Implicaciones teóricas}

Este estudio realiza varias contribuciones a la literatura. En primer lugar, corrobora la multidimensionalidad del P–J Fit, demostrando que el ajuste técnico y el ajuste socio–cultural predicen resultados diferenciados: el primero se asocia con actitudes hacia el trabajo (satisfacción), mientras que el segundo se vincula con evaluaciones de la relación con la organización (POS). Esta distinción respalda trabajos previos que separan el ajuste necesidades–suministros del ajuste valores–cultura \cite{lauver_distinguishing_2001,Cable2002}. En segundo lugar, se identifica al apoyo organizacional percibido como mediador crítico y a menudo dominante en la ruta ajuste $\rightarrow$ intención de renuncia, extendiendo los modelos tradicionales que privilegian la satisfacción como mediador principal \cite{tett_job_1993,griffeth_meta_2000}. En ocupaciones altamente técnicas, donde el capital humano es escaso, la reciprocidad organizacional emerge como mecanismo de retención más potente que la satisfacción en sí misma. En tercer lugar, el estudio integra de manera rigurosa distintos paradigmas psicométricos Teoría Clásica de los Tests, AFE/AFC y Teoría de Respuesta al Ítem—mostrando que la depuración basada en TRI puede mejorar sustancialmente el ajuste de los modelos SEM (CFI/TLI de 0.96 a 0.99) sin perder validez ni fiabilidad. Esta combinación metodológica aporta un protocolo replicable para optimizar instrumentos de medida en futuros estudios. En cuarto lugar, la investigación se desarrolla en un contexto latinoamericano poco estudiado (tecnólogos colombianos en sectores eléctrico/electrónico), aportando evidencias de la vigencia del modelo en este entorno y ofreciendo un instrumento validado de 15 ítems que facilita la evaluación de riesgos de rotación en prácticas de recursos humanos. Finalmente, los hallazgos sitúan al POS como variable clave para teorías de intercambio social y reciprocidad organizacional \cite{eisenberger_perceived_1986,rhoades_perceived_2002}, sugiriendo que la valoración del empleado como recurso crucial incrementa los sentimientos de obligación y reduce la rotación.

\subsection{Implicaciones prácticas}

Para organizaciones que buscan retener talento técnico especializado, los resultados implican tres líneas estratégicas:

\begin{enumerate}
\item \textbf{Optimizar el ajuste técnico desde la selección y el desarrollo}. Seleccionar candidatos cuyas competencias se alineen con los requisitos del puesto y ofrecer formación continua (reskilling/upskilling) garantiza que el \emph{Skill--Job Fit} se mantenga alto. El modelo mostró que un buen ajuste técnico potencia la satisfacción laboral y el POS, lo que indirectamente reduce la intención de renuncia. Prácticas como el análisis detallado del puesto, entrevistas conductuales basadas en competencias y programas de entrenamiento específicos refuerzan esta dimensión.
\item \textbf{Cultivar y visibilizar el apoyo organizacional}. El POS emergió como el mediador más fuerte y un predictor directo significativo de la permanencia. Es crucial que las organizaciones demuestren de manera tangible su reconocimiento al talento técnico, a través de recompensas equitativas, oportunidades de desarrollo profesional, feedback constructivo y un clima de apoyo supervisor. Estas acciones aumentan la percepción de que la organización valora y se preocupa por sus empleados, activando obligaciones de reciprocidad que disuaden la renuncia.
\item \textbf{Diseñar estrategias diferenciadas según tipo de ajuste}. Dado que el ajuste técnico influye principalmente a través de la satisfacción y el socio–cultural a través del POS, las intervenciones deben adaptarse a cada dimensión. Para maximizar la satisfacción, se recomienda ampliar la autonomía, la variedad y el significado del trabajo \cite{hackman_work_1980}, ajustando el diseño del puesto a las competencias del empleado. Para maximizar el POS, se deben fortalecer prácticas de apoyo socioemocional y reconocimiento público del valor de los empleados, así como fomentar una cultura de justicia y reciprocidad.
\end{enumerate}

En un mercado laboral donde la demanda por tecnólogos especializados supera la oferta, estas estrategias pueden ser decisivas para atraer y retener talento. Además, el instrumento abreviado de 15 ítems ofrece una herramienta práctica para diagnósticos rápidos de P–J Fit, satisfacción, POS e intención de renuncia, permitiendo a las organizaciones monitorear longitudinalmente la eficacia de sus políticas de recursos humanos.

\subsection{Limitaciones y líneas de investigación futura}

A pesar de sus aportes, este estudio presenta limitaciones. El diseño transversal impide establecer relaciones causales definitivas y no descarta la posibilidad de reciprocidad inversa (por ejemplo, que la intención de renuncia influya en la percepción de apoyo). El muestreo no probabilístico reduce la generalización de los resultados a otros contextos o sectores; sería deseable replicar la investigación con métodos de muestreo probabilístico y en otras regiones. Las mediciones se basan en autoinformes, lo que podría generar varianza común del método; futuras investigaciones deberían usar diseños multiinformante o integrar indicadores objetivos (por ejemplo, datos de permanencia real). Se recomienda explorar moderadores como la edad, la antigüedad y la disponibilidad de empleo en el sector, así como comparar resultados con otros tipos de ajuste (por ejemplo, ajuste persona–organización o persona–equipo). Finalmente, un diseño longitudinal permitiría evaluar la estabilidad temporal del P–J Fit y sus efectos, y captar con mayor precisión los mecanismos de mediación y la influencia de eventos organizacionales (promociones, recortes, cambios tecnológicos).



